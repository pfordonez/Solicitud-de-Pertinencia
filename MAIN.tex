%% Copyright 2006-2013 Xavier Danaux (xdanaux@gmail.com).
%
% Este trabajo puede ser distribuido o modificado bajo las 
% condiciones de la LaTeX Project Public License V1.3c, 


\documentclass[12pt,a4paper,sans]{moderncv}       
\moderncvstyle{oldstyle}                          
\moderncvcolor{blue}                              
\usepackage[utf8]{inputenc}      
\usepackage{titling}
\usepackage[spanish, english]{babel}			  
\usepackage[scale=0.75]{geometry}
\usepackage{setspace}
\usepackage{datetime}

% Información personal
\name{Nombres}{Apellidos}
\CI{1103674733} %Cédula de identidad
\newcommand{\titulo}{Título del Proyecto de Titulación}
\newcommand{\tituloEng}{Title}
\address{dirección}{código postal}{país}
\phone[mobile]{3115483925}
%\phone[fixed]{3115483925} 
\email{correoUNL@unl.edu.ec}                               
%\homepage{www.johndoe.com}  
%\extrainfo{información adicional}               %\quote{Some quote} % opcional, remover o comentar si no quiere una frase o cita

\begin{document}

\recipient{Ing. Pablo F. Ordoñez Ordoñez, Mg. Sc.}{Director de la CIS/C-UNL}
\date{Loja, 23 de marzo del 2021}
\opening{De mi consideración:}
\closing{En la seguridad de que su autoridad, se digne ordenar el trámite legal y reglamentario a la presente petición, anticipo mis más sinceros agradecimientos. Se adjunta la evaluación en PTT 2021}
\enclosure[Anexos]{Proyecto de Trabajo de Titulación, periodo abril-sep 2021, Evaluación del Tutor en PTT}          
\makelettertitle

De conformidad a lo preceptuado en los Arts: \textit{123} y \textit{128} del Reglamento de Régimen Académico de la Universidad Nacional de Loja, régimen 2009 y con la finalidad de optar el grado en Ingeniería en Sistemas de la Facultad de la Energía, las Industrias y los Recursos Naturales No Renovables, adjunto un ejemplar del Proyecto de Trabajo de Titulación, el mismo que versa sobre \textbf{\titulo} - \textbf{\tituloEng} a efecto de que el responsable de la línea de investigación, emita, para su pertinencia respectiva, un informe sobre la coherencia, estructura y pertinencia académica del proyecto; y, si está de acuerdo a su trascendencia, que amerite ser investigado.

\makeletterclosing

\end{document}